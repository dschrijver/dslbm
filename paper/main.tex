\documentclass[aps,physrev,twocolumn,10pt]{revtex4-2}

% Added myself
\usepackage[colorlinks=true,linkcolor=blue,citecolor=blue,urlcolor=blue]{hyperref}
\usepackage{bm}
\usepackage{amsmath}

\begin{document}

\title{A Generalized Non-Equilibrium Bounce-Back Condition for Multi-Component DdQn Lattice Boltzmann Models with Wetting Applications.}

\author{D. Schrijver}
\affiliation{}

\date{\today}

\begin{abstract}
Abstract will come here\dots
\end{abstract}

\maketitle

\section{\label{sec:Introduction}Introduction}
Introduction will come here\dots

\section{Knudsen expansion}
In diffusive scaling, we get the following for a Knudsen expansion,

\begin{equation}
\begin{split}
    f_i^{(0)} &= f_i^{\text{eq}, (0)} = w_i \rho^{(0)},\\
    f_i^{(1)} &= f_i^{\text{eq}, (1)} = w_i \rho^{(1)} + w_i \rho^{(0)} \frac{\bm{u}^{(1)} \cdot \bm{c}_i}{c_s^2},\\
    f_i^{(2)} &= f_i^{\text{eq}, (2)} - \frac{\tau}{\textit{Sh}}\left(c_{i\alpha}\partial_\alpha f_i^{\text{eq},(1)}\right).\\
\end{split}
\end{equation}
This gives the non-equilibrium part up to second order in Knudsen number,
\begin{equation}
\begin{split}
    f_i^{\text{neq}} &= -\frac{\tau}{\textit{Sh}}\Biggl(\partial_\alpha \Bigl(w_i c_{i\alpha}\rho^{(1)}\Bigr) + \frac{\rho^{(0)}}{c_s^2}\partial_\alpha \Bigl(w_i c_{i\alpha} c_{i\beta}u_\beta\Bigr)\Biggr)\\
    & \quad + \mathcal{O}\left(\textit{Kn}^3\right).
\end{split}
\end{equation}
We now want to plug this into the equation for the NEBB boundary condition,
\begin{equation}
\begin{split}
    f_i^{\text{neq}} &= f_q^\text{neq} - \sum_\alpha w_i c_{i\alpha}N_\alpha,\\
    N_\alpha &= \frac{1}{\sum^\text{in}_i w_i c_{i\alpha}^2}\sum^\text{par}_i f_i c_{i\alpha} \equiv C_\alpha \sum^\text{par}_i f_i c_{i\alpha}. 
\end{split}
\end{equation}
We now want to add stress corrections,
\begin{equation}
    f_i^{\text{neq}} = f_q^\text{neq} - \sum_\alpha w_i c_{i\alpha}N_\alpha - \sum_\mu \sum_\nu w_i c_{i\mu}c_{i\nu}M_{\mu\nu}
\end{equation}
The corrections can be found using the expected Navier-Stokes second moment,
\begin{equation}
    \Pi_{\alpha\beta}^{\text{NS}} = -\tau c_s^2 \rho \left(\partial_\alpha u_\beta + \partial_\beta u_\alpha\right) 
\end{equation}
The velocity gradients can be estimated using finite differences. We now want to ensure the following,
\begin{equation}
    \sum_i c_{i\alpha}c_{i\beta}f_i = \Pi_{\alpha\beta}^{\text{NS}}. 
\end{equation}
This gives the following tensor equation,
\begin{equation}\label{eq:matrix equation}
    \sum_\mu \sum_\nu Q_{\alpha\beta\mu\nu} M_{\mu\nu} = B_{\alpha\beta}
\end{equation}
With the following definitions,
\begin{equation}
\begin{split}
    Q_{\alpha\beta\mu\nu} &\equiv -\sum_i^\text{in}w_i c_{i\alpha}c_{i\beta}c_{i\mu}c_{i\nu},\\
    A_{\alpha\beta} &\equiv \Pi_{\alpha\beta}^{\text{NS}} - B_{\alpha\beta},\\
    B_{\alpha\beta} &\equiv C_{\alpha\beta} + D_{\alpha\beta}\\
    C_{\alpha\beta} &\equiv 2\sum^\text{out}_i c_{i\alpha}c_{i\beta}f_i + \sum^\text{par}_i c_{i\alpha}c_{i\beta}f_i \\
    &\quad + 2\rho\sum_i^\text{in}w_ic_{i\alpha}c_{i\beta} \frac{\bm{u} \cdot \bm{c}_i}{c_s^2}\\
    D_{\alpha\beta} &\equiv  - \sum_\gamma\sum_i^\text{in}w_ic_{i\alpha}c_{i\beta} c_{i\gamma}N_\gamma.
\end{split}
\end{equation}
For a boundary in the $yz$, $xz$ and $xy$-planes, Eq.~(\ref{eq:matrix equation}) can be solved if we use the D3Q27 stencil. The results are given in Eqns.~(\ref{eq:M_x}),~(\ref{eq:M_y}) and~(\ref{eq:M_z}). Now that we have $M_{\mu\nu}$, we can compute $N_\alpha$, 
\begin{equation}
\begin{split}
    \rho u_\alpha &= \sum_i c_{i\alpha}f_i\\
    &= \sum_i^\text{in}c_{i\alpha}\left(2 w_i \rho \frac{\bm{u}\cdot \bm{c}_i}{c_s^2}- \sum_\beta w_i c_{i\beta}N_\beta + m_\alpha\right)\\
    &\quad + \sum^\text{par}_i c_{i\alpha}f_i + \frac{1}{2}F_{\alpha},\\
    m_\alpha &= \delta_{n\alpha}B_{\alpha\alpha} + (1 - \delta_{n\alpha})2B_{n\alpha},
\end{split}
\end{equation}
where the index $n$ indicates the component normal to the boundary. We can now define the following,
\begin{equation}
\begin{split}
    D_{nn} &= -N_n / 6,\\
    D_{n\alpha} &= -N_\alpha/18
\end{split}
\end{equation}

\onecolumngrid
\newpage
\begin{equation}\label{eq:M_x}
    \underline{\underline{M}}^{yz} = \begin{pmatrix}
    -12 A_{00} + 9 A_{11} + 9 A_{22} & -18 A_{01} & -18 A_{02}\\
    -18 A_{01} & 9 A_{00} - 27 A_{11} & -54 A_{12}\\
    -18 A_{02} & -54 A_{12} & 9 A_{00} - 27 A_{22}
    \end{pmatrix}
\end{equation}

\begin{equation}\label{eq:M_y}
    \underline{\underline{M}}^{xz} = \begin{pmatrix}
    -27 A_{00} + 9 A_{11} & -18 A_{01} & -54 A_{02}\\
    -18 A_{01} & 9 A_{00} - 12 A_{11} + 9 A_{22} & -18 A_{12}\\
    -54 A_{02} & -18 A_{12} & 9 A_{11} - 27 A_{22}
    \end{pmatrix}
\end{equation}

\begin{equation}\label{eq:M_z}
    \underline{\underline{M}}^{xy} = \begin{pmatrix}
    -27 A_{00} + 9 A_{22} & -54 A_{01} & -18 A_{02} \\
    -54 A_{01} & -27 A_{11} + 9 A_{22} & -18 A_{12} \\
    -18 A_{02} & -18 A_{12} & 9 A_{00} + 9 A_{11} - 12 A_{22}
    \end{pmatrix}
\end{equation}
\twocolumngrid


\bibliography{references}

\end{document}
